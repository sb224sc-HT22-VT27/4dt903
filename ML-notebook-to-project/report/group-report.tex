\documentclass{article}
\usepackage{graphicx}
\usepackage[colorlinks=true,
  linkcolor=black,
  urlcolor=blue,
citecolor=black]{hyperref}
\usepackage{amsmath}
\usepackage{booktabs}
\usepackage{siunitx}
\usepackage{subcaption}
\usepackage{xcolor}
\usepackage{cite}
\usepackage[nottoc]{tocbibind}
\usepackage[english]{babel}

\tolerance=1413
\hfuzz=1.5pt

\title{4DT903 Group Report \\
\large{Model-Driven Development of Notebook-to-Project Transformation System}}
\author{Samuel Berg, Jesper Wingren \& Emil Ulvagården}
\date{January 2025}

\begin{document}

\maketitle

\tableofcontents

\newpage
\section{Introduction}

This report presents the collaborative work of our group on developing a model-driven engineering (MDE) solution for transforming Jupyter notebooks into structured project artifacts. The project addresses the challenge of converting exploratory data science notebooks into maintainable, production-ready code structures.

\subsection{Background}

Jupyter notebooks are widely used in data science for exploratory analysis, prototyping, and documentation. However, transitioning from notebooks to production-ready projects presents several challenges including code organization, modularity, testing, and maintainability.

\subsection{Project Objectives}

The main objectives of this project are:
\begin{itemize}
    \item Design metamodels for representing notebook structures and project architectures
    \item Implement model-to-model (M2M) transformations from notebook models to project structure models
    \item Develop model-to-text (M2T) transformations to generate project artifacts
    \item Create a comprehensive system demonstrating MDE principles in practice
\end{itemize}

\newpage
\section{Methodology}

\subsection{Model-Driven Engineering Approach}

Our approach follows the fundamental principles of MDE:
\begin{enumerate}
    \item \textbf{Metamodel Design}: Creating Ecore metamodels to represent the domain concepts
    \item \textbf{Model Transformations}: Implementing transformations using ATL/ETL or similar technologies
    \item \textbf{Code Generation}: Using template engines like Acceleo or Xtend for M2T transformations
\end{enumerate}

\subsection{Development Process}

The development followed an iterative approach with regular group meetings and collaborative decision-making on architectural choices.

\subsection{Tools and Technologies}

\begin{itemize}
    \item Eclipse Modeling Framework (EMF) for metamodel implementation
    \item Eclipse Modeling Tools for model editing and validation
    \item Transformation languages for M2M transformations
    \item Template engines for code generation
    \item Version control with Git for collaboration
\end{itemize}

\newpage
\section{Metamodels}

\subsection{Notebook Metamodel}

The notebook metamodel captures the essential structure of Jupyter notebooks:
\begin{itemize}
    \item Notebook container with metadata
    \item Cell structures (code cells, markdown cells)
    \item Cell dependencies and execution order
    \item Import statements and external dependencies
\end{itemize}

\subsection{Project Structure Metamodel}

The target metamodel represents a well-organized project structure:
\begin{itemize}
    \item Package/module organization
    \item Source file structure
    \item Configuration files
    \item Test organization
    \item Documentation structure
\end{itemize}

\subsection{Design Decisions}

Key design decisions in the metamodels include:
\begin{itemize}
    \item Abstraction level of the models
    \item Relationships between model elements
    \item Constraints and validation rules
    \item Extension points for customization
\end{itemize}

\newpage
\section{Transformations}

\subsection{Model-to-Model Transformation}

The M2M transformation maps notebook structures to project structures:
\begin{itemize}
    \item Grouping related code cells into modules
    \item Extracting function definitions
    \item Identifying test cases
    \item Organizing imports
\end{itemize}

\subsection{Model-to-Text Transformation}

The M2T transformation generates concrete project artifacts:
\begin{itemize}
    \item Python source files with proper structure
    \item Configuration files (requirements.txt, setup.py)
    \item Test files following conventions
    \item Documentation files (README.md)
\end{itemize}

\subsection{Transformation Patterns}

Common transformation patterns implemented:
\begin{itemize}
    \item Sequential code to function extraction
    \item Data processing pipelines to class methods
    \item Visualization code to utility functions
    \item Markdown cells to docstrings and documentation
\end{itemize}

\newpage
\section{Implementation}

\subsection{System Architecture}

The system consists of several Eclipse plugins:
\begin{itemize}
    \item NotebookMM: Core metamodel for notebook representation
    \item ProjectStructureMM: Target metamodel for project structure
    \item NotebookToProjectM2M: Transformation logic
    \item ProjectStructureM2T: Code generation templates
\end{itemize}

\subsection{Key Components}

\subsubsection{Metamodel Implementation}

The metamodels are implemented using Ecore with proper validation rules and constraints.

\subsubsection{Transformation Implementation}

Transformations leverage established patterns and best practices from the MDE community.

\subsubsection{Code Generation}

The code generation templates ensure generated code follows Python best practices and conventions.

\subsection{Testing and Validation}

\begin{itemize}
    \item Programmatic tests for transformation logic
    \item Example notebooks for validation
    \item Generated code quality checks
\end{itemize}

\newpage
\section{Results}

\subsection{Achieved Functionality}

The implemented system successfully:
\begin{itemize}
    \item Parses notebook structures into models
    \item Transforms notebook models to project structures
    \item Generates organized Python project code
    \item Maintains traceability between source and target
\end{itemize}

\subsection{Example Transformation}

% TODO: Add specific examples with input notebook and generated output

\subsection{Evaluation}

The system was evaluated based on:
\begin{itemize}
    \item Correctness of transformations
    \item Quality of generated code
    \item Completeness of the solution
    \item Maintainability of the models and transformations
\end{itemize}

\newpage
\section{Discussion}

\subsection{Strengths}

\begin{itemize}
    \item Clear separation of concerns through metamodeling
    \item Reusable transformation logic
    \item Extensible architecture
    \item Well-documented models
\end{itemize}

\subsection{Limitations}

\begin{itemize}
    \item Limited support for complex notebook patterns
    \item Manual intervention needed for certain transformations
    \item Scope limitations in handling all notebook features
\end{itemize}

\subsection{Challenges Encountered}

\begin{itemize}
    \item Balancing abstraction level in metamodels
    \item Handling diverse notebook structures
    \item Ensuring generated code quality
    \item Tool integration and setup
\end{itemize}

\subsection{Lessons Learned}

\begin{itemize}
    \item Importance of iterative metamodel design
    \item Value of transformation testing early in development
    \item Benefits of using established MDE tools and patterns
    \item Collaboration and version control in modeling projects
\end{itemize}

\newpage
\section{Conclusion}

\subsection{Summary}

This project successfully demonstrated the application of model-driven engineering principles to the practical problem of notebook-to-project transformation. The resulting system provides a foundation for automated code organization and can be extended to support additional transformation patterns.

\subsection{Contributions}

Each team member contributed to different aspects of the project:
\begin{itemize}
    \item Metamodel design and implementation
    \item Transformation development
    \item Code generation and templates
    \item Testing and validation
    \item Documentation
\end{itemize}

\subsection{Future Work}

Potential extensions and improvements:
\begin{itemize}
    \item Support for additional programming languages
    \item More sophisticated code analysis and refactoring
    \item Integration with CI/CD pipelines
    \item GUI for transformation configuration
    \item Machine learning-based transformation suggestions
\end{itemize}

\newpage
\section{References}

% Bibliography will be generated here if using BibTeX
% For now, add references manually as needed

\begin{thebibliography}{9}

\bibitem{emf}
Eclipse Modeling Framework (EMF),
\url{https://www.eclipse.org/modeling/emf/}

\bibitem{jupyter}
Project Jupyter,
\url{https://jupyter.org/}

\bibitem{mde}
Brambilla, M., Cabot, J., \& Wimmer, M. (2017).
\textit{Model-Driven Software Engineering in Practice}.
Morgan \& Claypool Publishers.

\end{thebibliography}

\end{document}
